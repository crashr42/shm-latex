\section{Тесты}
Рассмотрим тестирование на примере подачи заявки на регистрацию (листинг
\ref{lst:test_bid}).
При подаче заявке важно чтобы при создании, одобрении или отклонении заявки -
заявитель был уведомлен по email о соответствующем действии.

\begin{lstlisting}[language=Ruby,caption=Тестирование подачи заявки
,label={lst:test_bid}] 
require 'spec_helper'

describe Bid do
  before { BidMailer.deliveries.clear }

  it 'should be valid' do
    b = build(:bid)
    b.should be_valid
  end

  # #{Проверяем что после создания заявки, будет отослано письмо заявителю}
  it 'should send email after created' do
    b = create(:bid)
    BidMailer.deliveries.count.should eq(1)
    BidMailer.deliveries.last.to.should eq([b.email])
  end

  # #{Проверяем что после отклонения заявки будет отослано письмо заявителю}
  it 'should rejected' do
    b = create(:bid)
    BidMailer.deliveries.clear
    b.reject
    BidMailer.deliveries.count.should eq(1)
    BidMailer.deliveries.last.to.should eq([b.email])
    b.status.should eq('rejected')
  end

 # #{Проверяем что после одобрения заявки будет отослано письмо заявителю}
 it 'should approved' do
    b = create(:bid)
    BidMailer.deliveries.clear
    Role.stub(:find_by_name).with('patient').and_return(create(:patient_role))
    b.approve
    BidMailer.deliveries.count.should eq(1)
    BidMailer.deliveries.last.to.should eq([b.email])
    b.status.should eq('approved')
  end
end
\end{lstlisting}

Так как в системе используются параметры разного типа нужно тестировать правила
валидации для каждого параметра. В листинге \ref{lst:test_bool_parameter} тестируются
возможные значения для булевого параметра.

\begin{lstlisting}[language=Ruby,caption=Тестирование возможных значений для
булевого параметра ,label={lst:test_bool_parameter}] 
describe BoolParameter do
  context 'validate metadata' do
    it 'should be valid' do
      p = build(:bool_parameter, :metadata => {
          :values => %w(true false),
          :default => 'false'
      })
      p.should be_valid
    end

    it 'should be invalid' do
      p = build(:bool_parameter, :metadata => {
          :values => '',
          :default => ''
      })
      p.should_not be_valid
      p.errors[:metadata].should include('parameter.bool.metadata.errors.default')
      p.errors[:metadata].should include('parameter.bool.metadata.errors.values')
    end
  end

  context 'validate value' do
    before(:each) do
      @pr = build(:bool_parameter)
    end

    it 'should be valid' do
      @pr.validate_value(true).should eq(true)
      @pr.validate_value(false).should eq(true)
      @pr.validate_value('true').should eq(true)
      @pr.validate_value('false').should eq(true)
    end

    it 'should be not valid' do
      @pr.validate_value(Class).should eq(false)
      @pr.validate_value(123).should eq(false)
    end
  end
end
\end{lstlisting}

Более сложными тесты проверяют правильность обработки событий в системе.
Например событие нельзя сразу перевести и статуса free в close, так как
нарушается очередность состояний события. Тесты для проверки событий приведены в
приложении ?.
