\section{Диагностика}
\subsection{Прием данных}
Прием диагностических данных в системе осуществляется через отсылку запроса на
REST API. Запрос представляется из себя стандартный POST запрос по адресу
\url{http://localhost:3000/diagnostic/parameter} (листинг
\ref{lst:sending_diagnostic}) с указанием дополнительных параметров:
\begin{enumerate}
  \item user\_id - идентификатор пользователя в системе;
  \item parameter\_id - идентификатор мараметра;
  \item value - значение параметра.   
\end{enumerate}

\begin{lstlisting}[language=Bash,caption=Отправка диагностических данных
,label={lst:sending_diagnostic}] 
user@localhost$ curl -d "user_id=1&parameter_id=2&value=44" \
http://localhost/diagnostic/parameter
\end{lstlisting}

\subsection{Доступ к диагностическим данным}
Доступ к диагностическим данным предоставляется доктору. Доктор может
просматривать данные в виде графиков или таблиц. Графики формируются с помощью
класса ChartFactory в зависимости от класса параметра. ChartFactory имеет метод
build который принимет в качестве параметров:
\begin{enumerate}
  \item patient\_id - идентификатор пациента;
  \item parameter\_id - идентификатор параметра;
  \item from - начальная дата для выборки данных;
  \item to - конечная дата для выборки данных.
\end{enumerate}

Метод возвращает ассоциативный массив со структурой понятной \\ Highcharts.
После чего массив сериализуется в JSON и отдается клиенту.

\subsection{События}
После поступления диагностических данных в систему, системы инициирует
специальное событие. Событие указывает Websocket серверу оповестить всех
заинтересованных подписчиков о том что диагностические данные обновились.

Данный механизм позволяет доктору просматривать поступающие диагностические
данные в реальном времени.

Для доступа к диагностическим данным в реальном времени используется Websocket
клиент.