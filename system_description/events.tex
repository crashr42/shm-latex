\subsection{События}
В системе реализована событийная модель, характеризующая реальные процессы:
прием, обследование. Базовам классом для всех событий является класс Event.
Событие может находиться в 4 состояниях:
\begin{enumerate}
  \item free - событие доступо для работы (например на прием у врача со статусом
  free можно записаться); 
  \item busy - событие занято (например на прием к врачу со статусом busy
  записаться не получится); 
  \item process - событие обрабатывается (врач ведет прием пациента);
  \item close - событие завершено (прием окончен).   
\end{enumerate}

Статус события может меняться только в определенном порядке:
\begin{enumerate}
  \item free -> busy (запись на прием);
  \item busy -> free (освободить запись);
  \item busy -> process (начать прием);
  \item process -> close (завершить прием).
\end{enumerate}

Такой подход обусловлен тем что с каждым переходом может быть связано
определенное действие. Если не фиксировать возможные переходы, то для перехода,
например, со статуса free -> close нужно будет создавать дополнительный
обработчик.

Для контроля смены статуса событий и создания обработчиков этих переходов был
создан модуль Workflow. Реализация в виде модуля обусловлена тем что позволяет
внедрять данный модуль в любой класс. Пример использования модуля для обработки
переходов для событий приведен в листинге \ref{lst:using_workflow_module}.

\begin{lstlisting}[language=Ruby,caption=Использование модуля Workflow
,label={lst:using_workflow_module}] 
class Event < ActiveRecord::Base
  include Workflow

  # #{Возможные переходы для статуса события}
  workflow :status do
    flow :default, :busy => :process
    flow :default, :process => :close

    flow :reset_duration, :free => :busy do
      if self.event.present?
        self.event.duration -= self.date_end - self.date_start
      end
      self.duration = 0
    end

    flow :reset_duration, :busy => :free do
      if self.event.present?
        self.event.duration += self.date_end - self.date_start
      end
      self.duration = self.date_end - self.date_start
    end
  end
end
\end{lstlisting}