\subsection{Кабинеты}
Весь функционал системы распределен по личным кабинетам - самостоятельным
(реализация каждого модуля независима друг от друга) javascript-приложениям.
Данное распределение позволяет сделать систему понятной, безопасной, уменьшается
избыточность исходного кода. Общий принцип работы приложения на javascript
представлен на диаграммах последовательностей - приложения 
\ref{app:sequence_client}, \ref{app:sequence_server}.

После авторизации на главной странице, пользователь переходит по ссылке в свой
личный кабинет. Основная навигация происходит с помощью верхнего меню.

Доктор в своем кабинете может открыть список пациентов (приложение
\ref{app:doctor_cabinet_patients}) и кликнув на конкретного пациента в списке
получает возможность работать с его учетной записью:
просмотреть и указать диагноз (см. сценарий в приложении
\ref{app:doctor_cabient_diagnosis_assigning}), просмотреть список лекарств (приложение \ref{app:doctor_cabinet_medicament}) которые принимает пациент, назначить
пациенту врачебный прием, а также записать пациента на обследование к другому
доктору.

Одним из видов коммуникации между пациентом и доктором в системе является
врачебный прием (приложение \ref{app:doctor_cabinet_appointments}). Помимо
возможности записи на прием, в кабинете доктора есть страничка приема. В момент
когда пациент заходит в приемный кабинет, доктор (или его ассистент) должен
нажать кнопку “Начать прием”, при этом будет отмечено фактическое начало приема.
После этого в системе становится доступным отмена или назначение приема лекарств
пациенту. По окончании приема необходимо нажать кнопку “Завершить прием”. При
необходимости доктор составляет документ по результатам приема.
Сценарий приема пациента изображен в приложении
\ref{app:doctor_cabient_appointment}.

В кабинете пациента доступно расписание событий (приложение
\ref{app:patient_cabinet_events}), которые назначены пациенту.
Как правило это приемы у врача, лечебные процедуры, уведомления о приеме
лекарств.

Помимо просмотра событий, пациент в своем личном кабинете может записаться на
прием к своему лечащему врачу. Сценарий реализующий данную возможность приведен
в приложении \ref{app:patient_assigning_appointment_seq_diagr}

Так же доступен список лекарств и параметров назначенных пациенту (приложение
\ref{app:patient_cabinet_main}).
Есть возможность выбрать лечащего врача.

Основной функцией пользователя-пациента является дистанционная подача  значений
своих медицинских параметров в медицинское учреждение. Это производится путем
ввода информации в специальные веб-формы (приложения
\ref{app:patient_cabinet_parameters}, \ref{app:enter_parameter}). Доктор в своем личном
кабинете имеет возможность просматривать значений параметров указанные его пациентами, а также
может ознакомиться с результатами аналитики, которые представлены в виде таблиц,
графиков и диаграмм (приложение \ref{app:doctor_cabinet_diagnostic}).

В кабинете менеджера находится панель управления всеми пользователями системы.
Также одной из функций менеджера является рассмотрение заявок на регистрацию
нового пользователя в системе (приложение \ref{app:manager_cabinet_bid}).

Если пациент хочет принять участие в проекте, то он должен заполнить заявку
(приложение \ref{app:bid_form}) на регистрацию. В ней он указывает свои учетные
данные медицинского учреждения (номер страхового полиса, больничной карты), а
также прикладывает отсканированное изображение паспорта.

Менеджер системы просматривает поступившие заявки, проверяет верность указанных
в них данных и либо создает нового пользователя, либо отклоняет заявку.