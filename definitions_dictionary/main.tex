\section*{Словарь терминов и определений}
\addcontentsline{toc}{section}{Словарь терминов и определений}
\underline{Развертывание} - процесс переноса приложения на рабочий сервер и
последующий запуск приложения в рабочем режиме.

\underline{Issue tracking} - программное обеспечение для создания задач с
возможностями:
\begin{enumerate}
  \item отслеживать статус выполнения задач;
  \item комментировать задачи;
  \item соотносить изменения в коде с задачей.
\end{enumerate}

\underline{MVC} («Модель-представление-контроллер») - схема использования
нескольких шаблонов проектирования, с помощью которых модель данных приложения,
пользовательский интерфейс и взаимодействие с пользователем разделены на три
отдельных компонента так, что модификация одного из компонентов оказывает
минимальное воздействие на остальные.  Каждый из компонентов означает:
\begin{enumerate}
  \item Модель - предоставляет знания: данные и методы работы с этими данными, реагирует на запросы, изменяя своё состояние. Не содержит информации, как эти знания можно визуализировать.
  \item Представление, вид - отвечает за отображение информации (визуализацию).
Часто в качестве представления выступает форма (окно) с графическими элементами.
  \item Контроллер - обеспечивает связь между пользователем и системой:
контролирует ввод данных пользователем и использует модель и представление для реализации необходимой реакции.   
\end{enumerate}

\underline{ORM} (Object-relational mapping) - технология программирования,
которая связывает базы данных с концепциями объектно-ориентированных языков
программирования, создавая «виртуальную объектную базу данных».

\underline{REST} (Representational State Transfer) - «передача представлений
состояний».
Был предложен в 2000 году Роем Филдингом. Данные в REST должны передаваться в
виде небольшого количества стандартных форматов (например HTML, XML, JSON).
Сетевой протокол (как и HTTP) должен поддерживать кэширование, не должен
зависеть от сетевого слоя, не должен сохранять информацию о состоянии между
парами «запрос-ответ».

\underline{HTTP} (HyperText Transfer Protocol) -  протокол прикладного уровня
передачи данных (изначально — в виде гипертекстовых документов). Основой HTTP является
технология «клиент-сервер», то есть предполагается существование потребителей
(клиентов), которые инициируют соединение и посылают запрос, и поставщиков
(серверов), которые ожидают соединения для получения запроса, производят
необходимые действия и возвращают обратно сообщение с результатом.

\underline{XML} (eXtensible Markup Language) - рекомендованный Консорциумом
Всемирной паутины (W3C) язык разметки. Спецификация XML описывает XML-документы и частично
описывает поведение XML-процессоров (программ, читающих XML-документы и
обеспечивающих доступ к их содержимому). XML разрабатывался как язык с простым
формальным синтаксисом, удобный для создания и обработки документов программами
и одновременно удобный для чтения и создания документов человеком, с
подчёркиванием нацеленности на использование в Интернете. Язык называется
расширяемым, поскольку он не фиксирует разметку, используемую в документах:
разработчик волен создать разметку в соответствии с потребностями к конкретной
области, будучи ограниченным лишь синтаксическими правилами языка. Сочетание
простого формального синтаксиса, удобства для человека, расширяемости, а также
базирование на кодировках Юникод для представления содержания документов привело
к широкому использованию как собственно XML, так и множества производных
специализированных языков на базе XML в самых разнообразных программных
средствах.

\underline{JSON} (JavaScript Object Notation) - текстовый формат обмена данными,
основанный на JavaScript и обычно используемый именно с этим языком. Как и
многие другие текстовые форматы, JSON легко читается людьми.

\underline{HTML} (HyperText Markup Language) - стандартный язык разметки
документов во Всемирной паутине. Большинство веб-страниц создаются при помощи
языка HTML (или XHTML). Язык HTML интерпретируется браузерами и отображается в
виде документа в удобной для человека форме. HTML является приложением («частным
случаем») SGML (стандартного обобщённого языка разметки) и соответствует
международному стандарту ISO 8879. XHTML же является приложением XML.

\underline{CRUD} (Create Read Update Delete) - сокращённое именование 4 базовых
функций при работе с персистентными хранилищами данных — создание, чтение,
редактирование и удаление.

\underline{DDL} (Data Definition Language) - это семейство компьютерных языков,
используемых в компьютерных программах для описания структуры баз данных.

\underline{Websocket} - протокол полнодуплексной связи поверх TCP-соединения,
предназначенный для обмена сообщениями между браузером и веб-сервером в режиме
реального времени.

\underline{Hot Standby} - механизм поддержки состояния резервного компонента
системы в актуальном состоянии, позволяющий производить замену основного
компонента без задержки.

\underline{SMTP} (Simple Mail Transfer Protocol — простой протокол передачи
почты) — это сетевой протокол, предназначенный для передачи электронной почты в
сетях TCP/IP.

\underline{SSH} (Secure SHell) - сетевой протокол прикладного уровня,
позволяющий производить удалённое управление операционной системой и
туннелирование TCP-соединений (например, для передачи файлов). Схож по
функциональности с протоколами Telnet и rlogin, но, в отличие от них, шифрует
весь трафик, включая и передаваемые пароли. SSH допускает выбор различных
алгоритмов шифрования. SSH-клиенты и SSH-серверы доступны для большинства
сетевых операционных систем.

\underline{Unix domain socket} (Доменный сокет Unix) или IPC-сокет (сокет
межпроцессного взаимодействия) — конечная точка обмена данными, схожая с
Интернет-сокетом, но не использующая сетевой протокол для взаимодействия (обмена
данными). Он используется в операционных системах, поддерживающих стандарт
POSIX, для межпроцессного взаимодействия. Корректным термином стандарта POSIX
является POSIX Local IPC Sockets.

\underline{SSL} (Secure Sockets Layer) - криптографический протокол, который
обеспечивает безопасность связи через Интернет. Он использует асимметричную
криптографию для аутентификации ключей обмена, симметричное шифрование для
сохранения конфиденциальности, а коды аутентификации сообщений для целостности
сообщений.
