\newpage
\section*{Выводы}
\addcontentsline{toc}{section}{Выводы}
Основная цель исследований представленных в данной ипломной работе заключалась в
создании системы мониторинга состояния детей в врожденным пороком сердца.
Базой для исследований стала деятельность Кузбасского кардиологического центра.
Были формализованны текущие бизнес-процессы и выявлены проблемы их
функционирования. Основной проблемой оказалась невозможность постоянного
наблюдения за состоянием пациента.

На основе проблем были сформированы цели. Основной целью стала организация
постоянного мониторинга состояния пациента. 

Для достижения цели были составлены требования к будущей системе. Основным требования стало
создание технической базы, которая позволяла бы получать диагностические данные
от пациента в максимально удобной для пациента форме. Так же важной технической
возможностью системы является получение диагностических данных с медицинских
устройств.

На основе целей были сформированы требования к будущей системе. Требования
включают в себя как набор необходимых функциональных возможностей системы, так и
требования к технической реализации.

Ни одно готовое решение не подошло под составленные требования. Основными
причинами были: спорная техническая реализация, отсутствие части функционала,
высокая стоимость. Именно после этого шага было принято разрабатывать
собственную систему.

На начальном этапе разработки были скорректированны существующие
бизнес-процессы, для адаптации их к новым требованиям. Далее была спроектирована
основная архитектура системы. Было решено реализовывать всю функциональность на
основе web-технологий. Основной причиной стала высокая доступность и
распространенность этих технологий.

Основную сложность при разработке новой системы создал вопрос с выбором
конкретных технологий. Были рассмотрены и испробованы различные технические
решения. В итоге выбор был сделан в пользу Ruby On Rails, как решения
предоставляющего наилучшую инраструктуру для разработки.

После продолжительного этапа разработки была реализована основная архитектура
системы и функциональность. Однако на данном этапе система еще непригодна для
использования на реальном предприятии.

Исходный код текущей реализации доступен в публичном репозитории
(\url{https://github.com/crashr42/shm}).
