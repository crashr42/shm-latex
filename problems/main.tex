\newpage
\section{Проблемы}
В существующем бизнес-процессе существует ряд недостатков которые снижают
эффективность процесса лечения.

Процесс лечения и мониторинга детей с ВПС является достаточно длительным, сроки
измеряются годами. Обусловлен такой длительный период многими факторами,
рассмотрим основные из них.

\subsection{Задержка с операционным вмешательством}
Лечение врожденного порока сердца возможно только с помощью операционного
вмешательства, которое может задерживаться. Основной причиной задержки является
денежный вопрос, потому что операции детей с ВПС достаточно дорогостоящие
(средняя стоимость открытой операции на сердце — 236 000 рублей\footnote{
http://www.pomogi.org/projects/heart }).
Важно вести постоянный контроль за состоянием пациента в дооперационный период.

\subsection{Наблюдение в послеоперационный период}
Наблюдение в послеоперационный период очень важно из-за рисков осложнений и
возможности повторных операционных вмешательств.

\subsection{Расстояние}
Не в каждом городе есть специализированная клиника для лечения детей с ВПС.
Из-за задержки с операцией необходимо либо переезжать в другой город для того
чтобы лечащий врач мог контролировать состояние ребенка, либо периодически
приезжать на осмотр. Тот и другой способы достаточно затратны, и к тому же могут
негативно сказаться на состоянии ребенка.

\subsection{Взаимодействие}
В дооперационный и послеоперационный перид наблюдение за состоянием ребенка
ведет как правило кардиолог по месту жительства, а операцию проводит уже другой
врач-хирург. Как правило хирург и кардиолог непосредственно не контактируют друг
с другом. Предоставление возможностей общаться и делиться информацией о пациенте
в между хирургом и кардиологом в процессе лечения позитивно скажется на процессе
реабилитации и лечения.

\subsection{Анализ, прогнозирование, тенденции}
Выше было сказано что процесс лечения достаточно длителен. Важно хранить всю
историю лечения в одном месте с возможностью простого доступа к ней.

\subsection{Лечение в стационаре}
Длительное пребывание пациента в стационаре снижает его социальные навыки -
ребенок остается без общения со сверстниками, много времени проводит внутри
помещения, затрудняется активное времяпрепровождение (если оно возможно). Также
происходит отрыв ребенка от образовательного процеса, что очень влияет на его
дальнейшие жизненные достижения. В связи с этим, важно свести реабилитационный
период к минимуму.
