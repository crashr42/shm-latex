\documentclass[russian,utf8,landscape,a1paper,noignorestamp]{eskdgraph}
\usepackage[11pt]{moresize}
\usepackage{anyfontsize}



\ESKDdocName{\small{Разработка информационной системы автоматизации
удалённого мониторинга пациентов, нуждающихся в постоянном наблюдении на 
примере больных с ВПС}}
\ESKDscale{1:1}
\renewcommand{\ESKDcolumnXfIname}{Студент}
\ESKDauthor{Калесников Д.С.}
\renewcommand{\ESKDcolumnXfIIname}{Студент}
\renewcommand{\ESKDcolumnXfIIIname}{}
\ESKDchecker{Кошкин Н.Г.}
\renewcommand{\ESKDcolumnXfVname}{Руковод.}
\ESKDnormContr{Ванеев О.Н.}
\renewcommand{\ESKDcolumnXfVIname}{Зав. каф.}
\ESKDapprovedBy{Чичерин И.В}%  "Увт." в штампе на листе содержания
\ESKDdate{2013/05/18} % Дата (Год отображается на титульной странице)

\begin{document}
\ESKDstyle{formI}

\center{\fontsize{100}{110}\selectfont \bf Тема дипломной работы}
~\linebreak
~\linebreak
 
\center{\fontsize{80}{110}\selectfont Разработка
информационной системы автоматизация удалённого мониторинга пациентов, нуждающихся в постоянном наблюдении на 
примере больных с ВПС}

~\linebreak
~\linebreak
~\linebreak
~\linebreak

\center{\fontsize{100}{110}\selectfont \bf Цель}
~\linebreak
~\linebreak

\center{\fontsize{80}{110}\selectfont Повысить эффективность процесса
мониторинга здоровья пациентов, за счет предоставления возможности получать данные от пациента и устройств
мониторинга в автоматического режиме}

~\linebreak
~\linebreak
~\linebreak
~\linebreak
\center{\fontsize{100}{110}\selectfont \bf Задачи}

~\linebreak
~\linebreak

\begin{itemize} \fontsize{80}{90}\selectfont % начало
% помеченного списка
\item[] Организовать постоянный мониторинг состояния
пациента;% первая помеченная запись
\item[] Автоматизировать процесс амбулаторного
наблюдения;% вторая помеченная запись
\item[] Автоматизировать процесс стационарного
наблюдения;
\item[] Автоматизировать анализ собираемых
медицинских данных;
\item[] Автоматизировать взаимодействие пациента и
врача;
\end{itemize}

\end{document}