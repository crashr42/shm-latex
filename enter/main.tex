\newpage 
\subsection*{Введение}
\addcontentsline{toc}{section}{Введение}
В настоящее время, люди страдающие серьезными системными заболеваниями
(например, сердечно-сосудистыми) стали получать возможность проходить необходимое лечение и даже возвращаться (до
определенной степени) к полноценной жизни. Основная трудность с которой они
сталкиваются при этом - необходимость постоянного врачебного наблюдения с целью
сохранения достигнутого состояния оздоровления. Наблюдение предполагает собой
частые визиты к врачу; отсюда вытекает потеря личного времени пациента на
преодоление расстояния, на ожидание в очереди и др. Помимо этого на медицинское
учреждение накладывается функция сбора и анализа медицинской статистики.

Согласно исследованиям GBI Research\footnote{ http://ria-ami.ru/news/26944 } в
ближайшие годы здравоохранение столкнется с серьезными проблемами: повысится
доля пожилых граждан в общей структуре населения и значительно увеличится
численность пациентов с хроническими заболеваниями — сердечно-сосудистыми,
легочными, а также диабетом. По оценкам Всемирного фонда диабета, к 2025 г. 80\%
пациентов с диабетом будут проживать в странах, где подавляющее число граждан
обладают низкими или средними доходами.

На основе полученных результатов очевидно возрастание необходимости в удаленном
медицинском обслуживании. Технические средства удаленного мониторинга, с одной
стороны, избавляют пациентов от необходимости регулярно посещать лечащих врачей
(что особенно важно для обитателей удаленных регионов), а с другой — на
регулярной основе обеспечивают медицинских работников актуальной информацией о
состоянии здоровья их подопечных.

После внимательного анализа приведенных выше фактов, стала прояснятся общая
проблема, присущая данному рода медицинского обслуживания. Пациенту для
соблюдения непрерывного медицинского наблюдения необходимо личное присутствие в
медицинском учреждении, даже в самых малозначимых ситуациях.
В то же время, последние  несколько лет возросли темпы компьтеризации населения,
также повсеместно стало  распространяться относительно недорогое подключение к
сети Интернет. В связи с этим становится вполне логичной идея частично
реализовать общение пациента и врача с использованием современных информационных
технологий.

Таким образом, основной целью разработки  является создание такой системы,
которая бы позволила реализовать обмен медицинской информацией между доктором и
пациентом дистанционно, через сеть Интернет. Система также должна хранить
полученную информацию и выполнять типовые операции с ними с целью мониторинга.
В целях исследования и разработки системы нами были использованы бизнес-процессы
и организационная структура медицинского учреждения “Кузбасский кардиологический
центр”.