\newpage
\section{Структура системы}
Под структурой системы будем понимать совокупность элементов, составляющих
систему, и связи между ними. Ниже рассмотрим основные компоненты системы.
Деление в контексте структуры достаточно условное, но оно позволит разграничить
области ответственности каждого компонента системы. Разграничение
ответственностей позволит упростить структуру системы и снизить связанность
элементов системы. Каждая из подсистем должна соответствовать требованиям к
системе приведенным выше.

\subsection{Подсистема ввода данных}
Подсистема ввода данных должна соответствовать требованию “доступности на
запись” и обеспечивать слудующие возможности:

\begin{enumerate}
  \item Ручной ввод данных - непосредственно участниками системы, доктором,
  пациентом, менеджером.
  \item Автоматический ввод данных - получение данных с различных устройств
  диагностики состояния пациента.
  \item Проверка вводимых данных на корректность. Корректность данных
  определяется исходя из контекста использования данных и типа данных.
  \item Абстракция. Доступ к источнику данных должен быть через легкозаменяемую
  абстракцию. Это позволит не зависеть от конкретного поставщика и выполнить требование “гибкости”.
\end{enumerate}

\subsection{Подсистема доступа к данным}
Подсистема доступа к данным должна соответствовать требованию “доступности на
чтение” и обеспечивать уровень абстракции от источника данных чтобы
соответствовать требованию “гибкости”. Так же должен обеспечиваться доступ как
для внешних, так и для внутренних потребителей.

\subsection{Подсистема хранения данных}
Подсистемма хранения данных должна обеспечивать:
\begin{enumerate}
  \item Целостность данных
  \item Возможность управления доступом к данным
  \item Возможности ускорения доступа к данным
  \item Возможность компенсировать увелчение нагрузки
\end{enumerate}
Обязательными являются требования к “надежности”, “безопасности” и
“масштабируемости”.

\subsection{Подсистема анализа данных}
Подсистема анализа данных должна обеспечивать возможность анализа данных,
поступающих из подсистемы ввода данных. Доступ к данным осуществляется через
подсистему доступа к данным. Промежуточные результаты работы могут сохранятся в
подсистеме хранения данных.
Результаты работы подсистемы анализа данных должны быть представлены в виде двух
видов отчетов:

\begin{enumerate}
  \item Отчет по запросу
  \item Автоматический отчет
\end{enumerate}

\subsection{Подсистема управления доступом}
Подсистема управления доступом должна:
\begin{enumerate}
  \item Обеспечивать возможность контроля доступа к данным в зависисмости от роли пользователя в системе.
  \item Реагировать на попытки несанкционированного доступа к данным.
\end{enumerate}
