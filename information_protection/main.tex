\newpage
\ESKDthisStyle{formII}
\section{Информационная безопасность}
Существует определенный набор ГОСТ’ов, подробно описывающих источники угроз и
механизмы защиты от них. В данном разделе будет рассмотрена более простая
терминология. Б\'{о}льшее внимание будет уделено конкретным рекомендация по
обеспечению информационной бесзопасности в разрабатываемой системе.

\subsection{Основные угрозы}
Предже чем приступать к составлению рекомендаций нужно определиться с
терминологией.

Угрозой будем считать процесс, результатом которого является предоставление
несанкцианированного доступа к какой-либо информации.

Человека, объект или ПО целью которого является получение доступа к зашищенной
информации будем считать злоумышленником.

Будем рассматривать следующие виды угроз информационной безопасности:

\begin{enumerate}
  \item Первая – это аппаратные средства. Сбой в работе или выход из строя процессора, материнской платы, линий связи, периферийных устройств может привести к частичной или полной потери информации, хранящейся в компьютере.
  \item Второй источник угрозы – программное обеспечение. Угрозу могут
представлять исходные и приобретенные программы, утилиты и операционные системы. Для обеспечения сохранности информации штатным пользователям рекомендуется устанавливать лицензионные антивирусные программы.
  \item Третий вид угрозы информационной безопасности – данные, которые хранятся
на отдельных носителях или в печатном виде. Необходимо принимать отдельные меры по хранению данных, которые не находятся в компьютерной системе.
  \item Четвертый вид угрозы – пользователи компьютеров и обслуживающий
персонал. Люди могут нанести вред информации как случайно, так и специально. Поэтому всегда количество людей, имеющих доступ к информации организации, сведен до минимума. Большинство организаций открывают в штате должность специалиста по информационной безопасности, который отвечает за сохранность данных компьютерных систем.
\end{enumerate}

\subsection{Обеспечение безопасности}
\subsubsection{Аппаратный уровень}
Прежде всего нужно предотвращать физический доступ к серверам на которых
расположена важная информаци. В контексте разработанной системы - это любой
физический сервер непосредственно обеспечивающий работоспособность системы.

Так же важно резервировать основные компоненты системы и производить постоянное
архивирование важных данных для предотвращения потерь в случае сбоя в
оборудовании.

Давать более конкретные рекомендации для данного уровня не имеет смысла, так как
по большей части все зависит от конкретной схемы установки системы.

\subsubsection{Программный уровень}
На данном уровне обеспечение безопасности необходимо как на уровне
разрабатываемой системы, так и на уровне операционной системы и на уровне
обслуживающего програмного обеспечения.

Важно устанавливать програмное обеспечение только из доверенных источников;

На каждом физическом узле системы обеспечивать связь с другими узлами на
минимальном необходимом уровне. Прежде всего это значит закрытие все
неиспользуемых при работе системы портов. Конфигурация используемы портов
зависит от расположения компонентов системы и настроек системы. По-умолчанию в
системе используются следующие порты:
\begin{enumerate}
  \item 80 - Nginx;
  \item 5432 - Postgresql;
  \item 22 - ssh;
  \item 25 - SMTP;
  \item 8081 - Websocket server.   
\end{enumerate}

При использовании дополнительного программного обеспечения (бэкапы, логи) могут
использоваться:
\begin{enumerate}
  \item 9101, 9102, 9103 - Bacula;
  \item 514 - syslog.   
\end{enumerate}

Доступ к серверам по протоколу ssh должен быть разрешен только с компьютеров
ответственных лиц.

На каждом сервере должно быть настроено логирование всех действий, для
расследования сбоев системы и случаев несанкционированного доступа.

Любые кофигурационные файлы и настройки системы не должны располагаться в
публичном доступе.

По возможности необходимо обеспечить использование SSL протокола для доступа к
серверу приложений.

Для развертывания системы нужны права ни модификацию и изменение схемы базы
данных. Необходимо забирать данные привилегии после установки для предотвращения
несанкцианированного изменения схемы базы данных.

\subsubsection{Человеческий фактор}
Частично на уровне системы реализована защита от человеческого фактора. В
частности при длительном бездействии авторизованного пользователя будет
произведена блокировка аккаунта. Так же система предоставляет доступ
авторизованному пользователю только к определенной информации.

\subsubsection{Политика информационной безопасности}
Принятие ПИБ важный этап при организации безопасности системы. Политика должна
быть разработа соответствующим отделом или руководством предприятия с целью
повышения уровня защищенности информации.