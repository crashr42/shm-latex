\subsection{Подразделение связанное с предметной областью}

Поскольку цель нашей разработки является создание автоматизированой системы
мониторинга пациентов с ВПС, рассмотрим подразделение, которое занимается этим
вопросом.

Данным подразделением является Отделение детской кардиологии, которое входит в
состав Стационара ККЦ.

Центр детской кардиологии функционально объединяет стационарное и
поликлиническое звено. Основным направлением деятельности центра является
диагностика и подготовка к хирургическому лечению врождённых пороков сердца у
детей.

Для лечения детей с врождёнными пороками сердца  используются современные
методики:  выполнение  операций на открытом сердце в условиях искусственного
кровообращения  и эндоваскулярные малоинвазивные методики.

В ходе операций на открытом сердце устраняются врождённые пороки сердца с
преполнением малого круга кровообращения (дефект межжелудочковой перегородки,
дефект межпредсердной перегородки без чётких краёв, атриовентрикулярная
коммуникация), «синие» пороки (тетрада Фалло). Среди эндоваскулярных
вмешательств используются методики закрытия дефекта межпредсердной перегородки,
открытого артериального протока системой «Amplatzer».

В ходе работы центра постоянно происходит ротация врачебного персонала, что
позволяет наблюдать пациента с момента обращения в клинику и до момента оказания
хирургической коррекции, а так же осуществлять динамическое наблюдение в периоде
реабилитации.

Отделение рассчитано на 25 пациентов. Практическая работа осуществляется 10
сотрудниками. В штатах 4 врача детских-кардиологов, из которых 1 имеет высшую
категорию, 1 вторую квалификационную категорию, 6 медицинских сестёр, 3 с высшей
квалификационной категорией, 2 с первой.
