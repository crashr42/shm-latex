\subsection{История предприятия}
История создания Кузбасского кардиологического центра началась в марте 1957
года, когда в Кемеровской области была сделана первая операция на сердце -
пальцевая митральная комиссуротомия при митральном стенозе. Операцию проводил
заслуженный врач РФ, почетный гражданин города Кемерово, хирург М.А.
Подгорбунский на базе отделения торакальной хирургии Областной клинической
больницы №1.

Год спустя, осенью 1958 года был организован кабинет для ангиокардиографии. В
1974 году на основании приказа МЗ СССР «Об организации центра
сердечно-сосудистой хирургии в г. Кемерово» на базе Областной клинической
больницы № 1 открыто кардиологическое отделение на 40 коек, а с 1975 года - на
50 коек.

В 1989 году Администрация города Кемерово принимает решение о строительстве
Кемеровского кардиологического  испансера (ККД) на правом берегу реки Томи в
живописном сосновом бору. Организация такого специализированного учреждения была
вызвана необходимостью расширения диагностических и лечебных возможностей
кардиологической помощи больным, страдающим сердечно-сосудистыми заболеваниями.
Возглавил кардиодиспансер доктор медицинских наук, профессор, в настоящее время
академик РАМН Леонид Семенович Барбараш, один из пионеров кардиохирургии
Кемеровской области. Созданию и развитию кардиодиспансера активно помогали
руководители крупных промышленных предприятий, администрации города и области.

С 1994 года управление учреждением осуществляется двумя руководителями:
генеральным директором Цыганковой Галиной Юсифовной и главным врачом Барбарашом
Леонидом Семёновичем.

К 1994 году в ККД создана основная диагностическая и лечебная база. Это
амбулаторная служба (многопрофильная районная и специализированная
кардиологическая поликлиника), диагностические отделения (функциональной
диагностики, ультразвуковых исследований, лучевой диагностики, клиническая
лаборатория и др.) и стационарные отделения (острой коронарной патологии, общей
кардиологии, реабилитационное отделение, отделения сердечно-сосудистой хирургии
и реанимации). В составе кардиодиспансера активно развивались хозрасчетные
структуры, мобильный кардиологический диспансер, гараж, гостиница и пр.
   
В этот же период началось развитие научно - производственной базы, открыты
экспериментальная лаборатория, производство биопротезов клапанов сердца и
сосудов. В 2001 году создается Государственное учреждение
«Научно-производственная проблемная лаборатория реконструктивной хирургии
сердца и сосудов Сибирского Отделения Российской академии медицинских наук»
(ГУ НППЛ РХСС СО РАМН).

В августе 2005 года введен в эксплуатацию 12-ти этажный госпитальный корпус ККД,
что увеличило количество стационарных коек с 142 до 172. Открылись отделение
детской кардиологии, неврологическое, нейрохирургическое, значительно
увеличились объемы работы отделений сердечно-сосудистой хирургии и
рентгенхирургических методов диагностики и лечения.

С 2006 года ККД становится главным звеном медицинского комплекса «Кузбасский
кардиологический центр» совместно с ГУ НППЛРХСС СО РАМН и производством
биопротезов (ЗАО «Неокор»), обеспечивающий единый технологический цикл оказания
помощи пациентам при сердечно-сосудистых заболеваниях. Центр стал базой кафедры
кардиологии и сердечно-сосудистой хирургии КемГМА.

В декабре 2008 года ГУ НППЛРХСС СО РАМН реорганизуется в
Научно-исследовательский институт комплексных проблем сердечно-сосудистых
заболеваний Сибирского отделения РАМН, с большим научным потенциалом и хорошей
лечебно-диагностической базой.

В 2010г. Кемеровская область вошла в федеральную программу "Совершенствование
оказания медицинской помощи больным с острой сосудистой патологией". В рамках
реализации этой программы создан 1 региональный сосудистый центр (РСЦ) и 3
первичных сосудистых центра (ПСО). Базой РСЦ стал МУЗ "ККД". РСЦ -
координирующий головной центр в  регионе, оказывающий высокотехнологичную помощь
больным с сосудистыми заболеваниями. Созданы отделения для лечения больных с
острым нарушением мозгового кровообращения и острым коронарным синдромом.