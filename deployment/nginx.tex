\section{Nginx}
Архитектура работающего веб-сервера является двухуровневой. На первом уровне
находится HTTP-сервер, который перехватывает все HTTP запросы поступающие от
клиентов. В качестве такого сервера в нашем проекте используется бесплатный
сервер от Игоря Сысоева - nginx. Данный сервер длительное время он обслуживает
серверы многих высоконагруженных российских сайтов, таких как Яндекс, Mail.Ru,
ВКонтакте и Рамблер. Согласно статистике Netcraft nginx обслуживал или
проксировал 13.54\% самых нагруженных сайтов в мае 2013 года\footnote{
	\url{http://news.netcraft.com/archives/2013/05/03/may-2013-web-server-survey.html}
}.

Настройка сервера начинается с его установки. Это можно сделать обычными для
*nix систем способами - установить его через репозиторий пакетов, либо
скомпилировать из исходников с учетом особенностей конкретной рабочей машины.

Далее необходимо настроить конфигурацию для конкретного сайта (nginx позволяет
хостить множество сайтов). Для это в папке конфигурации надо создать файл с
именем сайта, как правило он расположен в папке
/etc/nginx/enabled-sites/site\_name.conf. В данном файле необходимо указать
полный URL сайта и номер порта. Также надо указать директорию в которой хранится
сайт. Как правило  для Ruby on Rails это /srv/site\_name/public.

С сервером второго уровня nginx связывается с помощью IPC-сокета, путь к 
которому указываются в конфигурационном файле сайта.