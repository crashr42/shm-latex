\subsection{Unicorn}
Сервер второго уровня получает поступающие запросы от сервера первого уровня,
выполняет их в среде Ruby on Rails, результат вычислений отдает в виде
стандартных веб-файлов (css, html, js) назад серверу первого уровня, который
отдает  их уже клиенту.

В качестве сервера второго уровня нами был выбран Unicorn. Данный сервер был
выбран за его популярность среди Ruby on Rails - разработчиков, что означает
наличие большого количества примеров файлов конфигурации, что ускоряет и
облегчает процесс развертывания.

Двухуровневая архитектура сервера дает определенные преимущества. При
использовании дополнительной библиотеки memcached можно закешировать в
оперативную память результаты вычислений сервера второго уровня. В этом случае
сервер первого уровня будет отвечать на запросы обращаясь к оперативной памяи,
что существенно ускорит ответ на запрос и разгрузит сервер.