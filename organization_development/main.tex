\newpage
\chapter{Организация процеса разработки}
\section{Определение условий разработки}
В связи с тем, что разработка проекта ведется в команде, необходимо решить
проблему совместного доступа к файлам проекта. Помимо этого, файлы проекта во
время работы постоянно претерпевают различные изменения. При этом часто бывает
важно иметь не только последние версии, но и несколько предыдущих. В простейшем
случае можно просто хранить несколько вариантов документа, нумеруя их
соответствующим образом. Такой способ неэффективен (приходится хранить несколько
практически идентичных копий), требует повышенного внимания и дисциплины и часто
ведёт к ошибкам, поэтому были разработаны средства для автоматизации этой
работы. В качестве решения данной проблемы неэффективности было решено
использовать системы управления версиями (VCS -  Version Control System).

\section{Система управления версиями Git}
Данная система  спроектирована как набор программ, специально разработанных с
учётом их использования в скриптах. Это позволяет удобно создавать
специализированные системы контроля версий на базе Git или пользовательские
интерфейсы.
Достоинствами данной системы являются:
\begin{enumerate}
  \item высокая производительность;
  \item децентрализованность;
  \item развитые средства интеграции с IDE;
  \item продуманная система команд.
\end{enumerate}

\section{Веб-сервис GitHub}\footnote{
	\url{https://github.com/}
} 
Это веб-сервис для хостинга проектов и их совместной разработки. GitHub
позиционируется как веб-сервис хостинга проектов с использованием системы
контроля версий git, а также как социальная сеть для разработчиков. Пользователи
могут создавать неограниченное число публичных репозиториев, для каждого из
которых предоставляется wiki, система issue tracking-а, есть возможность
проводить code review и многое другое. GitHub на данный момент является самым
популярным сервисом такого рода, обогнав Sourceforge и Google Code.

\section{Организация документации по проекту}
В процессе разработки проекта вместе с кодом создаются файлы документации. К ним
можно отнести документы разработки (записи требований заказчика, планы, отчеты о
ходе разработки), схемы, диаграммы и графические модели предметной области и пр.
В связи с этим возникает необходимость организации совместного доступа и
хранения данных файлов.

\subsubsection{Веб-приложение  Google Docs}
Данное приложение представляет из себя бесплатный онлайн-офис, включающий в себя
текстовый, табличный процессор и сервис для создания презентаций, а также
интернет-сервис облачного хранения файлов с функциями файлообмена,
разрабатываемый компанией «Google». Данный сервис также позволяет одновременное
совместное редактирование файлов.

\subsubsection{Веб-приложение diagram.ly}
Приложение позволяет создавать диаграммы различного типа. Поддерживает интеграцию с Google Docs.

\subsubsection{XMind}
XMind — это открытое программное обеспечение для проведения мозговых штурмов и
составления интеллект-карт, разрабатываемое компанией XMind Ltd.

Эта программа помогает пользователю фиксировать свои идеи, организовывать их в
различные диаграммы, использовать эти диаграммы совместно с другими
пользователями. XMind поддерживает интеллект-карты, диаграммы Исикавы (также
известные как fishbone-диаграммы или причинно-следственные диаграммы),
древовидные диаграммы, логические диаграммы, таблицы.

\subsubsection{Plant UML}\footnote{
	\url{http://plantuml.sourceforge.net/}
}
Plant UML - это открытое програмное обеспечение для постороения UML-диаграмм.
Важная особенность работы с Plant UML состоит в том, что любые диаграммы можно
описать на специальном языке в текстовой форме, после чего получить диаграмму в
виде png или svg файла.
