\subsection{Frontend}
Front-end - часть программы, которая взаимодействует с пользователем. Здесь мы
рассмотрим технологии используемые для построения графического интерфейса.

\subsubsection{Backbone.js}
Backbone.js придает структуру веб-приложениям с помощью моделей с биндингами по
ключу и пользовательскими событиями, коллекций с богатым набором методов с
перечислимыми сущностями, представлений с декларативной обработкой событий; и
соединяет это все с существующим REST-овым JSON API\footnote{
	\url{http://backbonejs.ru/}
}.

При использовании backbone.js данные предметной области представляются как
Модели (Models), которые могут быть созданы, провалидированы, удалены, и
сохранены на сервере. Всякий раз, когда в интерфейсе изменяется атрибуты модели,
модель вызывает событие "change"; все Представления (Views), которые отображают
состояние модели, могут быть уведомлены об изменении атрибутов модели, с тем
чтобы они могли отреагировать соответствующим образом — например, перерисовать
себя с учетом новых данных.

Основной полезный эффект возникающий от добавления backbone.js в проект
заключается в том, что разработчику не надо писать код, ищущий элемент с
определенным id в DOM и обновляющий HTML вручную. При изменении модели
представление просто обновит себя самостоятельно.

\subsubsection{Coffeescript}
Встроенная поддержка CoffeeScript была добавлена в Rails с версии 3.1. Программы
написанные на данном языке перед выполнением компилируются в javascript. Язык
CoffeeScript позволяет писать программы в функциональном стиле, в нем более
полно реализовано использование классов.

CoffeeScript используется чтобы улучшить читаемость кода и уменьшить его размер.
В среднем для выполнения одинаковых действий на CoffeeScript требуется в 2 раза
меньше строк, чем JavaScript\footnote{
	\url{http://coffeescript.org/}
}.

\subsubsection{RequireJs}
При разработке приложений с модульной структурой на JavaScript возникает две
проблемы:
\begin{enumerate}
  \item описание и удовлетворение зависимостей различных частей приложения, необходимость организации подключения зависимостей на серверной стороне;
  \item экспорт переменных в глобальную область видимости и их коллизия. 
\end{enumerate}

Озвученные проблемы можно решить используя фреймворк RequireJs. В этом случае на
странице достаточно использовать только один тег <script>. Все остальные js
файлы и библиотеки подключаются при вызове главной функции define. Пути к
подключаемым файлам передаются данной функции в качестве аргументов, а
возвращаемым значением будет являться весь javascript-контекст веб-страницы.

Подключаемым файлам необходимо назначить уникальное имя, по которому к нему
будет происходить обращение в результирующей функции (define).

\subsubsection{Twitter Bootstrap}
Для быстрой разработки интерфейса хорошо зарекомендовала себя библиотека (UI
Framework) Twitter Bootstrap. Framework содержит набор стилей CSS и javascript
функций, а также регламентирует варианты html разметки страницы: таблицы,
кнопки, стикеры, уведомления и многое другое. Для использования библиотеки
достаточно подключить несколько css стилей и javascript файлов. Далее при
создании веб-страниц для использования данной библиотекой для используемых тегов
достаточно указать необходимые значения атрибута class. Таблицу со значениями
атрибутов можно найти на официальном сайте\footnote{
	\url{http://twitter.github.io/bootstrap/}
}.

\subsubsection{Ресурсы приложения}
В Ruby on Rails все стили, скрипты js, картинки хранятся в папке app/assets. В
Ruby on Rails скрипты пишутся на языке coffee-script, а стили на SASS. В рабочем
режиме (production) исходные коды на этих языках компилируются в обычные CSS и
javacript файлы и затем на все входящие запросы отдаются как статичные файлы,
непосредственно веб-сервером. Благодаря такому подходу снижается нагрузку на
серверную машину, а следовательно уменьшается время отклика. В режиме
разработчика (development) перекомпиляция происходит при каждом запросе, для
оперативного просмотра изменений в исходном коде в процессе разработки.

\subsubsection{Средство построения графиков}
Основной целью разрабатываемой информационной системы является мониторинг
состояния здоровья пациентов. Основным средством визуального отображения
результатов мониторинга являются информационные графики и диаграммы. Для вывода
графиков используется javascript библиотека Highcharts\footnote{
	\url{http://www.highcharts.com/}
}.
