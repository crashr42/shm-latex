\section{План разрабоки}\footnote{
	\url{https://github.com/crashr42/shm/issues/milestones}
}
Процесс разработки разбит на несколько фаз. Фаза состоит из предварительного
набора задач по завершению которых  фаза будет считаться завершенной.

\subsection{Skeleton}
Цель - настройка среды для разработки, построение простейшего "скелета" системы:
\begin{enumerate}
  \item установка и настройка ruby, rvm, rails;
  \item установка и настройка postgres;
  \item инициализация проекта;
  \item настройка авторизации;
  \item установка админки.
\end{enumerate}

\subsection{General}
Цель - разрабока основы предметной области, доработка каркаса приложения:
\begin{enumerate}
  \item отражение основных объектов предметной области в виде классов моделей;
  \item покрытие тестам;
  \item организация стурктуры для js клиета.
\end{enumerate}

\subsection{Patient}
Цель - реализовать кабинет пациента:
\begin{enumerate}
  \item профиль;
  \item запись на прием;
  \item рассписание приемов у врача;
  \item рассписание приема лекарств;
  \item рассписание ввода показателей здоровья.   
\end{enumerate}

\subsection{Manager}
Цель - реализовать кабент менеджера:
\begin{enumerate}
  \item просмотр заявок;
  \item подтверждение заявок;
  \item отклонение заявок.   
\end{enumerate}

\subsection{Patient/Doctor}
Цель - организация взаимодействия между доктором и пациентом:
\begin{enumerate}
  \item общение пациента с доктором;
\end{enumerate}

\subsection{Doctor}
Цель - разработка кабинета доктора:
\begin{enumerate}
  \item просмотр своих пользователей;
  \item электронный прием;
  \item электронная запись на прием;
  \item назначение лекарств;
  \item назначение диагнозов;
  \item визуализация данных за период времени;
  \item отчеты.
\end{enumerate}

