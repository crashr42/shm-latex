\subsection{Тестирование}
Для контроля верности выполнения бизнес-процессов в проекте используется unit
тестирование с помощью библиотеки RSpec. Такой подход позволяет иключать
логические ошибки без полноценного запуска системы, как следствие повышается
скорость разработки.

Для удобства и автоматизации тестирования применяется концепция автоматического
тестирования. При каждом изменении в коде, если для данного изменения есть тест,
тест запускается и выводится уведомление о результате выполнения теста. Для
организации данного подхода используется библиотека Autotest.

Еще один ньюанс который нужно учитывать при тестировани заключается в том что
Ruby On Rails окружение запускается достаточно долго из за этого в несколько раз
увеличивается время выполнения тестов. Для решения данной проблемы используется
библиотека Spork. Spork запускает окружение Ruby On Rails и исключает
необходимость перезапускать окружение каждый раз. Так же Spork автоматически
перезагружает классы при изменении их исходного кода.