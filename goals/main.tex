\newpage
\section{Цели}
\subsection{Постоянный мониторинг состояния пациента	}

Постоянный мониторинг позволит получать наиболее актуальную информацию о
состоянии пациента в процессе лечения и во время реабилитационного периода. Так
же важно организовать ненавязчивый мониторинг в течении повседневной жизни
пациента. Рассмотрим основные направления мониторинга которые будут охвачены в
системе.

\subsubsection{Амбулаторное наблюдение}

Система должна позволять пациентам в добровольном порядке и в ненавязчивой форме
предоставлять данные осостоянии своего здоровья. Так как данные будут приходить
в систему из внешних незашищенных источников - необходимо обечпечивать
максимальную защищенность каналов передачи данных.

\subsubsection{Наблюдение в стационаре}

Необходимо организовать круглосуточное наблюдение за больными, помещенными в
специально оборудованное медицинское учреждение. В систему должны поступать
данные:

\begin{enumerate}
	\item с медицинских устройств;
	\item данные по результатам обследования;
	\item данные по результатам приемов и обходов.
\end{enumerate}

\subsection{Постоянный анализ получаемых данных}

Недостаточно просто хранить все данные в процессе лечения и возлагать
ответственность за их обработку на врача. Необходимо организовать обработку
данных в автоматическом режиме. Это позволит снизить нагрузку на врача и
повысить его эффективность в процессе лечения.
Реализация автоматическо обработки диагностических данных - достаточно сложный
процесс, поэтому ограничимся следующими направлениями в анализе данных:

\begin{enumerate}
	\item оценка эффективности лечения:
		\begin{enumerate}
			\item оценка влияния лекарственных препаратов;
			\item оценка влияния процедур.
		\end{enumerate}
	\item полный жизненный цикл процесса лечения.
\end{enumerate}

\subsection{Постоянное взаимодействие пациента с врачом}

Для повышения эффективности лечения пациента необходимо снизить издержки со
стороны пациента и врача на процесс общения и ибмена информации между ними.
Основным видом взаимодействия пациента и врача является личный прием у врача.
Такая форма взаимодействия наиболее эффективна и привычна с социальной и
профессиональных точек зрения, но она не всегда приемлима. В некоторых ситуация,
когда доктору или пациенту важно лишь уточнить некторые детали, лучше
организовать более простую форму взаимодействия между ними. Упрощенными формами
личного приема у врача могут являться:

\begin{enumerate}
  	\item интернет-прием - процесс представляющий из себя обычный прием у врача
организованный по средстам сети Интернет;
	\item  online-консльтация - процесс получения
унтересующих пациента сведений у специалиста в определенной области или
консультанта.
\end{enumerate}
 
Введение даных видов взаимодействия позволит в значительной мере сократить
нагрузку на врача и снизить временные и денежные издержки для пациента.

\subsection{Взаимодействие между врачами}

В процессе лечения пациента принимает участие широкий круг специалистов. Каждый
специалист должен иметь возможность получить в кратчайщие сроки информацию о:

\begin{enumerate}
	\item текущем состоянии пациента;
	\item заключениях других докторов;
	\item обследованиях и лекарстенных препаратах назначенных пациенту.
\end{enumerate}

Своевременное получение актуальной информации позволит более эффективно
организовать процесс лечения, за счет снижения временных затрат как пациента,
так и доктора.
