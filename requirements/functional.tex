\subsection{Функциональные требования}
\subsubsection{Пациент}
Данная роль является основной в разрабатываемой системе. Пользователи с данной
ролью будут иметь доступ к своим медицинским данным, возможность просмотра и
изменения (в рамках установленных границ) своего расписания, возможность
общаться с лечащим доктором, просмотр медицинских заключений, выданных доктором.
Также пользователи могут иметь возможность просмотра новостных рассылок сайта.

Основные варианты использования системы:

\begin{enumerate}
  \item расписание:
  \begin{enumerate}
    \item время приема лекарств;
    \item даты обследований;
    \item даты приемов у врача;     
  \end{enumerate}
  \item регистрация в системе - процесс регистрации в ситеме состоит из
  следующих этапов:
  \begin{enumerate}
    \item заполнение и подача электронной заявки на регистрацию. Подать заявку
    (на даном этапе анализа моделирования) могут только пациенты, проходящие
    лечение в Кузбасском кардиоцентре. 
    В заявке необходимо указать ФИО пациента и его матери (отца или опекуна), номер сотового телефона (для отправки на него аутентификационных данных), номер медицинской карточки, придуманный пользователем пароль;
    \item получение отказа или подтверждение на регистрацию в системе. В случае
    успешной регистрации пользователь получит аутентификационные данные для
    входа в систему. В атуентификационные данные входить сгенерированный логин и
    оставленный пользователем пароль.
  \end{enumerate}
  \item ввод показателей о состоянии здоровья согласно расписанию составленному
лучащим врачом пациента. Список показателей для мониторинга также составляется
врачом индивидуально для каждого пациента;
  \item электронная запись - процесс записи пациента к врачу, на обследование,
процедуры и другие услуги предоставляемы лечащим заведением с целью получать
актуальные данные о состоянии здоровья пациента и оперативно вносить изменения в процесс лечения или мониторинга;
\end{enumerate}






