\subsection{Требования к технологиям}
Надежность - способность системы сохранять работоспособность при нормальных
условиях эксплуатации.

Доступность - возможность свободного (разумеется, при наличии необходимых прав
доступа к системе) получения требуемой услуги 

Актуальность - соответствие
функциональности системы современным требованиям предполагаемой целевой
аудитории 

Поддержка - необходима дистанционная поддержка пользователей по
вопросам возникшим в результате работы системы. Данное требование должно быть
обязательно к исполнению в контексте предметной области (некоторые медицинские
процессы не требуют отлагательства). Также желательна возможность относительно
оперативного добавления или изменения текущего функционала системы.

Открытость - открытый доступ к системе, заключающийся в соблюдении
международных и национальных стандартов в области используемых информационных
технологий с целью свободного взаимодействия  программных приложений, данных,
персонала и пользователей системы.

Низкая стоимость - при исполнении данного требования желательно использование
программного обеспечения с открытым исходным кодом. Аппаратное обеспечение
должно без проблем поддерживать озвученные выше требования к системе, поэтому
для снижения расходов предпочтительно привлечение спонсоров.

Функциональность (специфика бизнеса, стратегические приоритеты, географическая
распределенность и т.д.)
