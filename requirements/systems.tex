\section{Требования к системе}

\subsection{Надежность}
\subparagraph{Поддержка целостности данных.}
Данные о пациенте будут хранится достаточно долгий промежуток времени в течении которого важно обеспечивать целостность данных. Под целостностью данных прежде всего понимаются:

\begin{enumerate}
  \item после поступления в систему данных из внешних систем, данные не должны
  менять своего состояния;
  \item целостность связей между данными. 
\end{enumerate}

\subparagraph{Резервирование основных узлов системы.}
Важно обеспечить доступность системы даже при отказе одного из узлов. Данное
qтребование может быть выполнено за счет дублирование основных узлов системы,
или распределения нагрузоки между однотипными узлами.

\subsection{Безопасность}
\subparagraph{Защита персональных данных больного.}

В соответствии с Законом № 152-ФЗ персональными данными является любая
информация, связанная с физическим лицом (субъектом персональных данных),
позволяющая идентифицировать конкретное физическое лицо среди прочих лиц. В
персональных данных физического лица выделяют общие и специальные категории.
Согласно данному закону, персональные данные это любая информация, относящаяся к
определенному или определяемому на основании такой информации физическому лицу
(субъекту персональных данных), в том числе его фамилия, имя, отчество, год,
месяц, дата и место рождения, адрес, семейное, социальное, имущественное
положение, образование, профессия, доходы, другая информация.
Среди конфиденциальной информации можно выделить медицинскую (или врачебную)
тайну. Российское законодательство определяет врачебную тайну как «информацию о
факте обращения за медицинской помощью, состоянии здоровья гражданина, диагнозе
его заболевания и иные сведения, полученные при его обследовании и лечении».
Фактически, на текущий момент защита личных данных в медицинских информационных
системах представлена  двумя базовыми аспектами.
Первым из них является этический (профессиональный) аспект взаимодействия врача
и пациента, который регулируется нормами врачебной этики и законом о защите
личных данных пациентов.
Второй аспект представляет собой защиту информации в медицинской системе с
технической точки зрения, то есть, здесь речь идет о создании адекватных
механизмов защиты данных непосредственно в рамках программно-аппаратного
комплекса информационной системы.
По мнению экспертов Фрайбургского университета (Германия), до 60\%\footnote{
	http://www.cnews.ru/reviews/free/national2006/articles/datasecure/
} утечек
медицинской информации происходит из-за действий медицинских работников, причем,
не только лечащих или консультирующих врачей, но и обслуживающего и
административного персонала медучреждений. Только 40\% утечек информации
происходит по техническим причинам — в результате взломов информационных систем
злоумышленниками, хищения баз данных и персональных компьютеров.

\subsection{Доступность}
\subparagraph{Доступность на чтение.}
Система должна быть доступна на чтение с любого устройства поддерживающего
доступ к сети интернет.
\subparagraph{Доступность на запись.}
Доступность ситемы на запись должна ограничиваться на уровне распределения прав
доступа к системе согласно ролям пользователей.

\subsection{Масштабируемость}
Масштабируемость - возможность системы справляться с возрастающими нагрузками за
счет модернизации системы. Важно понимать что масштабируемость должна
обеспечивать модернизацию системы с минимальными изменениями.

\subsection{Гибкость}
\subparagraph{Простота модернизации.}
Данное требование включает в себя как простоту обновления существующих
компонентов так и максимально быструю возможность расширения системы.

Обновление компонентов системы не должно быть критичным. Система должна
поддерживать так называемое “обновление на лету”. В идеале время неработоспособности системы при обновлении должно стремится к нулю.

Расширение функционала системы не должно приводить к существенной переработке
существующего фугкционала.

\subparagraph{Минимум зависимостей.}
Любая информационная система состоит из большого числа компонентов. Важно чтобы
связи между компонентами были минимальны. Выполение данного условия позволит
сделать систему более независисмой от конкретных технологий и технических
решений.