\newpage
\section{Готовые решения}
Тема разработки программного обеспечения и информационных систем для медицинских
учреждений в последнее время получила большое распространение. Многие
разработчики решают начать делать свой так называемый “стартап”, также часто
можно встретить предложения от ИТ-компаний.

Существующие решения можно разделить на несколько основных классов.

\subsection{Решения на базе системы 1С:Предприятие}
\subsubsection{1С Медицина Поликлиника}
Данное решение\footnote{
	\url{http://www.v8.1c.ru/solutions/product.jsp?prod_id=149}
} 
предназначено для автоматизации деятельности медицинских
организаций различных организационно-правовых форм, оказывающих медицинскую
помощь в амбулаторно-поликлинических условиях. Программный продукт служит для
ведения взаиморасчетов с контрагентами, управления потоками пациентов,
персонифицированного учета оказанной медицинской помощи.

\subsubsection{1С Рарус Амбулатория}
Данный продукт\footnote{http://rarus.ru/press/publications/126187/} комплексно
автоматизирует деятельность медицинского учреждения.
Помимо глубоко реализованной системы автоматизации документооборота, хотелось бы
отметить характерное для мира 1С систем наличие реестров и справочников
служебной медицинской информации, например, Банк Стволовых Клеток «КриоЦентр».

\subsection{Решения для автоматизации медицинского документооборота}
Комплексная медицинская информационная система (КМИС). Уменьшает затраты доктора
на ведение документации связанной с приемом пациентов, выдачей направлений и
т.д.
Медицинская информационная система AKSi-офис\footnote{
	\url{http://www.aksimed.ru/products/aksi_line/AKSi-Office.php}
} 
(на базе системы Microsoft Office).
Программное обеспечение от фирмы ТрастМед - аналогичный функционал.

\subsection{Комплексная автоматизация медицинского предприятия}
В первую очередь хотелось бы отметить отечественную разработку - Медицинская
информационная система AKSi-клиника от АКСИМЕД. Среди ее основных функций
хотелось бы отметить следующие:

\begin{enumerate}
  \item комплексная автоматизация всех процессов наблюдения, диагностики и
  лечения амбулаторных и стационарных пациентов;
  \item эффективное управление персоналом, ресурсами и
  финансово-\\экономической деятельностью ЛПУ, автоматизация
  медико-\\статистического контроля и планирования;
  \item однократный ввод информации в электронную историю болезни (электронную
  медицинскую карту) пациента с последующим многократным использованием этих сведений и поддержкой принятия врачебных решений;
  \item сквозная компьютеризация работы регистратуры, поликлиники, стационара,
  отделения скорой медицинской помощи, стоматологических кабинетов и других подразделений ЛПУ;
  \item обеспечение безопасности персональных данных в соответствии с
  Федеральным законом от 27 июля 2006 г. № 152-ФЗ.
\end{enumerate}

Также существуют множество зарубежных решений. Среди них можно упомянуть систему
разработанную и используемую в США - Practicefusion, чей девиз “Больше пациентов
- меньше работы”.

\subsection{Выбор готового решения}
Среди всех рассмотренных выше систем можно выявить общую тенденцию - ИТ-компании
предлагают в первую очередь автоматизацию медицинского документооборота.
Некоторые системы предлагают анализ и диагностику, но она заточена под широкое
использование.

Разработанная нами система позволяет решить поставленные в начале исследования
проблемы, а именно автоматизированный дистанционный (с определенной степенью)
сбор медицинской информации, мониторинг (проведение какого-либо анализа над
собранными), позволяет проводить коммуникацию между пациентами.

Разумеется озвученные нами возможности реализованы в существующих ныне системах
в том или ином виде. Здесь нужно отметить, что рассмотренные выше программные
средства и системы стоят больших денег и медицинское учреждение может сэкономить
внедряя нашу систему именно на том что наш система реализует конкретные
возможности (необходимые прежде всего для лечения больных с ВПС), а не внедряя
большой пакет возможностей, многие из которых могут никогда не пригодится.

Кроме того использование широко тиражированного программного обеспечения может
поставить в зависимость от решений фирмы-разработчика, что может оказать
нежелательным для предприятия со столь ответственной деятельностью. Наша же
система готова к дальнейшим изменениям потенциального заказчика, так
разрабатывается скорее для конкретных учреждений, а не для массовой реализации.